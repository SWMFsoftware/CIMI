The ring current can have a significant impact on the solution in the 
magnetosphere. It controls the energy density in the inner magnetosphere and
is important for the overall solution. 
Because of the importance of including 
such connections in our model, we now turn to the issue of how to use the CIMI
model as a component of the SWMF. 

\section{Configuring the SWMF to use CIMI}
Configuring the SWMF to use the CIMI is straight forward. Simply type:
\begin{verbatim}
  Config.pl -v=IM/CIMI
\end{verbatim}
Doing so tells the SWMF that the CIMI will represent the IM component of the 
SWMF. This configuration allows the user to execute the CIMI model through the 
SWMF without any other components.

Should the user wish to include, say, a global magnetosphere or the 
ionosphere electrodynamics the configuration is simply:
\begin{verbatim}
  Config.pl -v=IM/CIMI,GM/BATSRUS,IE/Ridley_serial
\end{verbatim}
Coupling with these other components is described in greater detail in the 
next section.

\section{Coupling the IM component with other components}
The CIMI model represents the IM component of the SWMF. It can be directly coupled 
to the Global Magnetosphere (GM) and Ionosphere Electrodynamics (IE) components 
by using the
\begin{verbatim}
  #COUPLE1
\end{verbatim}
 or
\begin{verbatim}
  #COUPLE2
\end{verbatim}
commands (See the SWMF manual for details). The IE component is coupled 
in a unidirectional manner with the IM component, by which the polar cap 
potential is passed to IM but no information 
flows in the opposite direction. (actually there is a feedback of currents provided, but it is not recommended). 

The IM-GM coupling is more complicated. The GM component can be coupled 
unidirectionally, with the IM component setting the densities and temperature at 
the IM outer boundary, or it can be set bidirectionally where the 
magnetospheric pressure and density is set from the IM model. 

There are several options for how the coupling between GM and IM can proceed. 
They all, however, start with the same command in the PARAM file:
\begin{verbatim}
  #COUPLE2
  GM            COMP1
  IM            COMP2
  10.0          DTcoupling
\end{verbatim}
The default method for coupling the IM and GM components is ...
In the case where composition is important, the GM component can be configured to 
use the multi-species equations:
\begin{verbatim}
  Config.pl -o=GM:e=MhdPw
\end{verbatim}
In this case the individual species densities are preserved, but the 
velocities are combined into a single fluid velocity. This is because the 
multi-species MHD representation of BATS-R-US solves separate continuity 
equations, but a single momentum and energy equation. 

The user also has the option to preserve velocity information as well as 
density information when putting the PW output into GM. This requires the 
user to configure the GM component to solve the full multi-fluid 
equations:
\begin{verbatim}
  Config.pl -o=GM:e=MultiIon
\end{verbatim}
In this case the individual species densities and velocities are conserved, as 
BATS-R-US solves a full continuity, momentum, and energy equation for each 
species.

The ansiotropic coupling can be activiated by using the following configuation
\begin{verbatim}
  Config.pl -o=GM:e=
\end{verbatim}
