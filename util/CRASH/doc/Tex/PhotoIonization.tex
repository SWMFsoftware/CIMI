%  Copyright (C) 2002 Regents of the University of Michigan, portions used with permission 
%  For more information, see http://csem.engin.umich.edu/tools/swmf
\section{Photoionization and photorecombination: effect of the Fermi statistics.}
The formula for the photoionization cross-section from atom (ion) in the charge state, $\zeta$, and with the electron configuration, $i$, reads:
\begin{equation}\label{eq:ion}
\sigma^{ion}_{\zeta,i\rightarrow \zeta+1,j} = \sigma_0 \frac{w_\eta n_\eta}{(\zeta+1)^2}\left(\frac{E_{ij}}{h\nu}\right)^3.
\end{equation}
Here 
\begin{equation}
\sigma_0 =\frac{64\pi}{3\sqrt{3}}\alpha a_0^2\approx 7.9\cdot 10^{-22}[{\rm m}^2],
\end{equation}
is the near-threshold ($E_{ij}=h\nu$) semi-classical cross-section of the photoionization 
of the hydrogen atom from the ground state 
($\zeta = 0$, $w_\eta = 1$, $n_\eta = 1$), which may be found in \cite{ZR}
\begin{equation}\label{eq:fine}
\alpha = \frac{e^2}{\hbar c}\approx 1/137.04
\end{equation}
is a fine structure constant, and
\begin{equation}\label{eq:a0}
a_0 = \frac{\hbar^2}{m_e e^2}\approx 0.5\cdot 10^{-10}[\rm m]
\end{equation}
is the Bohr radius. Eqs.(\ref{eq:fine},\ref{eq:a0}) are only valid in CGSE system of 
units, however, with the known numerical values for them all the
other formulae become insensitive to the choice of units. $h\nu$ is the energy of the 
photon, which is absorbed in the course of the photoionization. It should exceed
the transition energy,
\begin{equation}
h\nu\ge E_{ij}.
\end{equation}

The transition energy equals:
\begin{equation}
E_{ij} = I_\zeta-E_{\zeta,i}+E_{\zeta+1,j},
\end{equation}
where the excitation energy, $E_{\zeta,i}\ge0$ is introduced, which is the non-negative 
difference between the energy of the given electron configuration with respect to 
the ground state energy for the atom (ion) with the charge number equal to $\zeta$.

We consider only the transition to states $j$, such that the electron confiruration in 
these states are obtained from the initial configuration, $i$, by removing a single 
electron from $\eta$th electron orbit. $w_\eta$ and $n_\eta$ in Eq.(\ref{eq:ion}) mean 
the number of electron on this orbit and its principal quantum number correspondingly.

However, if the photoionization occurs in a plasma and the effects of the Fermi-
statistics in this plasma are not negligible, Eq.(\ref{eq:ion}) should be 
revisited. Recall, that the probabilities of the photon absorption processes in Quantum 
ElectroDynamics (QED) are calculated as follows (see \cite{lp}): (1) first, the matrix 
element of perturbation is calculated using the wave functions of bound and free 
electrons in initial and final states; (2) and then the matrix element should be 
integrated over all possible value of the free electron momentum, ${\bf p}$, as long as there is a 
free electron in the final state. In this integration the number of free electron 
states per the unit of space volume is introduced as follows:
\begin{equation}\label{eq:free}
N_{\bf p}=2\frac {dVd^3{\bf p}}{(2\pi\hbar)^3}.
\end{equation}
To account for the Fermi statistics effect on the electron gas we should note that:
(1) the calculation of the matrix element does not change, because in any case the wave function 
for a free electron is a plane wave; (2) however, the way to calculate the number of states
of free electrons change specifically. Specifically, we find that among the states as in 
Eq.(\ref{eq:free}) some places are already occupied, with the occupation numbers equal to
$1/(\exp[(\varepsilon_e -\mu)/(k_BT)]+1)$ (see \cite{ll}). Here $\mu$ is a chemical potential of
a free electron gas and $\varepsilon({\bf p})$ is its non-relativistic kinetic energy of the electron.
With this account, the actual number of free electron states is
\begin{equation}
N^{corr}_{\bf p} =2
\frac{dV d^3{\bf p}_e}{(2\pi\hbar)^3}
\left(
1-\frac1{\exp[\frac{\varepsilon_e -\mu}{k_BT}]+1}\right)=N_{\bf p}
\frac{\exp[\frac{\varepsilon_e -\mu}{k_BT}]}{\exp[\frac{\varepsilon_e -\mu}{k_BT}]+1}.
\end{equation}
Therefore, the corrected cross-section becomes:
\begin{equation}
\sigma^{corr,ion}_{\zeta,i\rightarrow \zeta+1,j} = \sigma^{ion}_{\zeta,i\rightarrow \zeta+1,j} 
\frac{\exp[\frac{\varepsilon_e -\mu}{k_BT}]}{\exp[\frac{\varepsilon_e -\mu}{k_BT}]+1}.
\end{equation}

The electron energy here is related to the photon energy, $h\nu$, via the conservation law:
\begin{equation}\label{eq:cons}
\varepsilon_e = h\nu - E_{ij}
\end{equation}

The contribution from the bound-free transition to the absorption coefficient 
(not yet corrected for the stimulated emission) may be now written in terms of the corrected cross-section:
\begin{equation}\label{eq:finalab}
\kappa^{bf}_{\nu} = \sigma_0\sum_{\zeta,i,j}{N_{\zeta,i} \frac{w_\eta n_\eta}{(\zeta+1)^2}\left(\frac{E_{ij}}{h\nu}\right)^3\frac{\exp[\frac{h\nu - E_{ij} -\mu}{k_BT}]}{\exp[\frac{ h\nu - E_{ij} -\mu}{k_BT}]+1}},
\end{equation}
where $N_{\zeta,i}$ is the concentration of ions with the given electron configuration.

Now, the cross-section of a photorecombination of an electron on a collision 
with the ion in the state, $\zeta +1,j$ is related to the cross-section as in Eq.(\ref{eq:ion}) in the following way:
\begin{equation}\label{eq:crossinv}
\sigma^{rec}_{\zeta+1,j\rightarrow\zeta,i}=
\frac{g_{\zeta,i} } { g_{\zeta+1.j} }\frac{(h\nu)^2} {2m_ec^2}
\frac{1}{\varepsilon_e}
\sigma^{ion}_{\zeta,i\rightarrow \zeta+1,j}=
\frac{g_{\zeta,i}p_p^2}{g_{\zeta+1.j} p_e^2}\sigma^{ion}_{\zeta,i\rightarrow \zeta+1,j},
\end{equation}
where $p_p=\varepsilon_p/c$ and $\varepsilon_p=h\nu$ are the momentum and the energy of a photon correspondingly.

The rule as in Eq.(\ref{eq:crossinv}) expresses the so-called {\it cross-invariance} property of all QED processes, which claims the possibility to relate the probability of
the direct and inverse processes if we accordingly account for the weights of the initial and final states. This property comes from the point that the matrix 
elements for the direct and reverse processes are exactly the same, the difference in the cross-sesctions occurs when the matrix element is integrated over the final 
states, which are different, because the final and initial states in the cross-invariant processes are changed by places. We emphasize, that the relationship as in 
Eq.(\ref{eq:crossinv}) holds with the {\it uncorrected} cross-section  $\sigma^{ion}_{\zeta,i\rightarrow \zeta+1,j}$, because it is based on the {\it uncorrected} 
statistical weight of free electron as in Eq.(\ref{eq:free}), which results in the multiplier, $p_e^2$, in the denominator of (\ref{eq:crossinv}). If we account the 
correcting factors both in $\sigma^{ion}_{\zeta,i\rightarrow \zeta+1,j}$ and in (\ref{eq:free}) these factors cancel each other and the expression for  
$\sigma^{rec}_{\zeta+1,j\rightarrow\zeta,i}$ keeps unchanged.

Integrate $(p_e/m_e)\sigma^{rec}_{\zeta+1,j\rightarrow\zeta,i}$ with the distribution function, 
$f_{\bf p}$, of an ideal Fermi gas of electrons,
\begin{equation}
f_{\bf p}d^3{\bf p}_e=\frac{2}{\exp[\frac{\varepsilon_e -\mu}{k_BT}]+1}\frac{d^3{\bf p}_e}{(2\pi\hbar)^3},
\end{equation}
and represent the result of integration by the electron momentum directions in the following form.
\begin{equation}\label{eq:emissivity}
\int{\frac{8\pi}{c^2 h^3}\sigma_0
\frac{g_{\zeta,i} } { g_{\zeta+1.j} }
\frac{w_\eta n_\eta}{(\zeta+1)^2}
\frac1{h\nu}
\frac{ E_{ij}^3 d\varepsilon_e}
{\exp[\frac{\varepsilon_e -\mu}{k_BT}]+1}}.
\end{equation}
Now discuss the physical meaning of the integrand as in (\ref{eq:emissivity}). In accordance with Eq.(\ref{eq:cons}), $d\varepsilon_e=d\varepsilon_p=d(h\nu)$. Hence the 
integrand in (\ref{eq:emissivity}) may be interpreted as the probability of emission, per a unit time interval, from an ion in the given state, with the emitted photon
being within the energy interval $d(h\nu) = d(E_{ij} + \varepsilon_e)$. Once been multiplied by $(h\nu)$, this becomes the emitted energy and once multipled by the emitters 
density, $N_{\zeta+1,j}$, it becomes the volumetric emissivity due to recombination, per a unity interval of the photon energies, ${\cal E}_\varepsilon$:
\begin{equation}\label{eq:finalem}
{\cal E}^{bf}_\varepsilon =\frac{8\pi}{c^2 h^3} \sigma_0
\sum_{\zeta,i,j}{N_{\zeta+1,j} \frac{g_{\zeta,i} } { g_{\zeta+1.j} }\frac{w_\eta n_\eta}{(\zeta+1)^2}
\frac{ E_{ij}^3}
{\exp[\frac{h\nu - E_{ij} -\mu}{k_BT}]+1}}.
\end{equation}

Discuss briefly the interaction of plasma with the euqilibrium black-body radiation. The absroption of the black-body radiation
(again, our Eq.(\ref{eq:finalab}) is not yet corrected for a stimulated emission) balances the emissivity, if the following condition is satisfied:
\begin{equation}
{\cal E}^{bf}_\varepsilon =\frac{8\pi(h\nu)^3\exp[-(h\nu)/(k_BT)]}{c^2 h^3} \kappa^{bf}.
\end{equation} 
Comparing this with Eqs.(\ref{eq:finalab},\ref{eq:finalem}), we find that the condition is satisfied, if:
\begin{equation}
 \frac{g_{\zeta,i} N_{\zeta+1,j}} {g_{\zeta+1.j} N_{\zeta.i} } = \exp[\frac{ -E_{ij} -\mu}{k_BT}] = \exp[\frac{ -I_\zeta +E_{\zeta,i}-E_{\zeta+1,j} -\mu}{k_BT}].
\end{equation}
This is exactly the expression relating the populations of the electron states in ions, which are in a thermodynamic equilibrium with a Fermi gas of electrons (see Part 1).
\section{Free-Free opacity}
The analisys of free-free absorption is mostly based on the same considerations as that for bound-free absorption. The distinction is in the use of the Fermi-Dirac 
distribution function not only for a free electron in the final state (what we did in the previous section) but also for the electron in the initial state.

The emission from an electron having the velocity, $v_E$ (again the subscript E means "emitter", subscript A means "absorber"), is given by Eq.(5.11) from \cite{RL}:
\begin{equation}
\frac{dE_\varepsilon}{dt}=\frac{32\pi^2e^6}{3\sqrt{3}c^3hm_e^2v_E}N_e\sum_\zeta{N_\zeta\zeta^2G_ff}
\end{equation}  

\section{Spectral temperature and group spectral temperatures}
In \cite{ll} it was mentioned that for any known spectral energy distribution , $E_\varepsilon$ at any value of $\varepsilon$, the local 
spectral temperature, $T^{Spectral}(E_\varepsilon(\varepsilon),\varepsilon)$ may be introduced, such that the spectral energy density 
$E_\varepsilon(\varepsilon)$ is locally equal to the Planckian spectral energy density at the temperature $T^{Spectral}$:
\begin{equation}
E_\varepsilon(\varepsilon)=E_\varepsilon^{(Pl)}(T^{Spectral},\varepsilon).
\end{equation}
Of course, if the radiation spectrum differs from that for equilibrium black-body radiation, the values of $T^{Spectral}$ are different at 
different energies, generally speaking.

In other words, for non-equilibrium spectrum, the distribution of spectral temperature may be used instead of the spectral energy density, as the spctrum characteristics.
The relationship between the spectral temperature and spectrum energy density is one-to-one mapping. For a given spectral temperature, the spectral energy density is given
by Eq.(\ref{eq:mg12}). For a given spectral energy density the spectral temperature amy be found using Eq.(63.26) from \cite{ll} which in our denotations may be written as 
follows:
\begin{equation}\label{eqmg14}
T^{Spectral}=\frac{\varepsilon/k_B}{\log\left(1+\frac{8\pi}{h^3c^3}\frac{\varepsilon^3}{E_\varepsilon}\right)}
\end{equation} 
Now we introduce the discrete analog of a spectral temperature. Specifically, we introduce {\it a group temperature}, $T_g$, such that the spectral energy of a black-body 
radiation, once integrated over the photon energy within a given group, would equal the given value of $E_g$:
\begin{equation}\label{eqmg15}
E_g(T_g)=\int_{\varepsilon_{g-1/2}}^{\varepsilon_{g+1/2}}{E_\varepsilon^{(Pl)}(T_g,\varepsilon)d\varepsilon}=\alpha T_g^4\frac{15}{\pi^4}\int_{\varepsilon_{g-1/2}/(k_BT_g)}^{\varepsilon_{g+1/2}/(k_BT_g)}{\frac{x^3dx}{\exp(x)-1}}.
\end{equation}
We will also need an expression for the group specific heat of the radiation, per a unit of volume:
\begin{equation}
C_g=\frac{dE_g}{dT_g}=4\alpha T_g^3\frac{15}{4\pi^4}\int_{\varepsilon_{g-1/2}/(k_BT_g)}^{\varepsilon_{g+1/2}/(k_BT_g)}{\frac{x^3\exp(x)dx}{[\exp(x)-1]^2}}.
\end{equation}
The procedure which allows us to obtain the radiation energy density per group and radiation specific heat per group from the given group temperature is 
straightforward. However, the inverse procedure is not that simple, because Eq.(\ref{eqmg14}) now only holds as an approximate one:
\begin{equation}\label{eq:mg16}
T^g\approx\frac{
\sqrt{\varepsilon_{g-1/2}\varepsilon_{g+1/2}}/k_B}
{
\log\left(
1+\frac{\alpha\varepsilon_{g-1/2}^2\varepsilon_{g+1/2}^2}{k_B^4}\frac{15}{\pi^4}\frac{\Delta(\log \varepsilon)}{E_g}\right)
}
\end{equation}
To satisfy Eq.(\ref{eqmg15}), one needs to improve the accuracy using Newton-Rapson procedure:
\begin{equation}
T_g^{n+1}=T_g^n+ (E_g-E_g(T_g^n))/C_g(T_g^n).
\end{equation} 

TBC
\section{Averaging opacities}
TBC
\section{Reduction of the multi-group diffusion equations to symmetric system of N temperature diffusion equations}
TBC
